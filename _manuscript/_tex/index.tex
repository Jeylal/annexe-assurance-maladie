% Options for packages loaded elsewhere
% Options for packages loaded elsewhere
\PassOptionsToPackage{unicode}{hyperref}
\PassOptionsToPackage{hyphens}{url}
\PassOptionsToPackage{dvipsnames,svgnames,x11names}{xcolor}
%
\documentclass[
]{agujournal2019}
\usepackage{xcolor}
\usepackage{amsmath,amssymb}
\setcounter{secnumdepth}{5}
\usepackage{iftex}
\ifPDFTeX
  \usepackage[T1]{fontenc}
  \usepackage[utf8]{inputenc}
  \usepackage{textcomp} % provide euro and other symbols
\else % if luatex or xetex
  \usepackage{unicode-math} % this also loads fontspec
  \defaultfontfeatures{Scale=MatchLowercase}
  \defaultfontfeatures[\rmfamily]{Ligatures=TeX,Scale=1}
\fi
\usepackage{lmodern}
\ifPDFTeX\else
  % xetex/luatex font selection
\fi
% Use upquote if available, for straight quotes in verbatim environments
\IfFileExists{upquote.sty}{\usepackage{upquote}}{}
\IfFileExists{microtype.sty}{% use microtype if available
  \usepackage[]{microtype}
  \UseMicrotypeSet[protrusion]{basicmath} % disable protrusion for tt fonts
}{}
\makeatletter
\@ifundefined{KOMAClassName}{% if non-KOMA class
  \IfFileExists{parskip.sty}{%
    \usepackage{parskip}
  }{% else
    \setlength{\parindent}{0pt}
    \setlength{\parskip}{6pt plus 2pt minus 1pt}}
}{% if KOMA class
  \KOMAoptions{parskip=half}}
\makeatother
% Make \paragraph and \subparagraph free-standing
\makeatletter
\ifx\paragraph\undefined\else
  \let\oldparagraph\paragraph
  \renewcommand{\paragraph}{
    \@ifstar
      \xxxParagraphStar
      \xxxParagraphNoStar
  }
  \newcommand{\xxxParagraphStar}[1]{\oldparagraph*{#1}\mbox{}}
  \newcommand{\xxxParagraphNoStar}[1]{\oldparagraph{#1}\mbox{}}
\fi
\ifx\subparagraph\undefined\else
  \let\oldsubparagraph\subparagraph
  \renewcommand{\subparagraph}{
    \@ifstar
      \xxxSubParagraphStar
      \xxxSubParagraphNoStar
  }
  \newcommand{\xxxSubParagraphStar}[1]{\oldsubparagraph*{#1}\mbox{}}
  \newcommand{\xxxSubParagraphNoStar}[1]{\oldsubparagraph{#1}\mbox{}}
\fi
\makeatother


\usepackage{longtable,booktabs,array}
\usepackage{calc} % for calculating minipage widths
% Correct order of tables after \paragraph or \subparagraph
\usepackage{etoolbox}
\makeatletter
\patchcmd\longtable{\par}{\if@noskipsec\mbox{}\fi\par}{}{}
\makeatother
% Allow footnotes in longtable head/foot
\IfFileExists{footnotehyper.sty}{\usepackage{footnotehyper}}{\usepackage{footnote}}
\makesavenoteenv{longtable}
\usepackage{graphicx}
\makeatletter
\newsavebox\pandoc@box
\newcommand*\pandocbounded[1]{% scales image to fit in text height/width
  \sbox\pandoc@box{#1}%
  \Gscale@div\@tempa{\textheight}{\dimexpr\ht\pandoc@box+\dp\pandoc@box\relax}%
  \Gscale@div\@tempb{\linewidth}{\wd\pandoc@box}%
  \ifdim\@tempb\p@<\@tempa\p@\let\@tempa\@tempb\fi% select the smaller of both
  \ifdim\@tempa\p@<\p@\scalebox{\@tempa}{\usebox\pandoc@box}%
  \else\usebox{\pandoc@box}%
  \fi%
}
% Set default figure placement to htbp
\def\fps@figure{htbp}
\makeatother





\setlength{\emergencystretch}{3em} % prevent overfull lines

\providecommand{\tightlist}{%
  \setlength{\itemsep}{0pt}\setlength{\parskip}{0pt}}



 


\usepackage{booktabs}
\usepackage{caption}
\usepackage{longtable}
\usepackage{colortbl}
\usepackage{array}
\usepackage{anyfontsize}
\usepackage{multirow}
\usepackage{float}
\usepackage{tabularray}
\usepackage[normalem]{ulem}
\usepackage{graphicx}
\usepackage{rotating}
\UseTblrLibrary{siunitx}
\NewTableCommand{\tinytableDefineColor}[3]{\definecolor{#1}{#2}{#3}}
\newcommand{\tinytableTabularrayUnderline}[1]{\underline{#1}}
\newcommand{\tinytableTabularrayStrikeout}[1]{\sout{#1}}
\usepackage{url} %this package should fix any errors with URLs in refs.
\usepackage{lineno}
\usepackage[inline]{trackchanges} %for better track changes. finalnew option will compile document with changes incorporated.
\usepackage{soul}
\linenumbers
\makeatletter
\@ifpackageloaded{caption}{}{\usepackage{caption}}
\AtBeginDocument{%
\ifdefined\contentsname
  \renewcommand*\contentsname{Table of contents}
\else
  \newcommand\contentsname{Table of contents}
\fi
\ifdefined\listfigurename
  \renewcommand*\listfigurename{List of Figures}
\else
  \newcommand\listfigurename{List of Figures}
\fi
\ifdefined\listtablename
  \renewcommand*\listtablename{List of Tables}
\else
  \newcommand\listtablename{List of Tables}
\fi
\ifdefined\figurename
  \renewcommand*\figurename{Figure}
\else
  \newcommand\figurename{Figure}
\fi
\ifdefined\tablename
  \renewcommand*\tablename{Table}
\else
  \newcommand\tablename{Table}
\fi
}
\@ifpackageloaded{float}{}{\usepackage{float}}
\floatstyle{ruled}
\@ifundefined{c@chapter}{\newfloat{codelisting}{h}{lop}}{\newfloat{codelisting}{h}{lop}[chapter]}
\floatname{codelisting}{Listing}
\newcommand*\listoflistings{\listof{codelisting}{List of Listings}}
\makeatother
\makeatletter
\makeatother
\makeatletter
\@ifpackageloaded{caption}{}{\usepackage{caption}}
\@ifpackageloaded{subcaption}{}{\usepackage{subcaption}}
\makeatother
\usepackage{bookmark}
\IfFileExists{xurl.sty}{\usepackage{xurl}}{} % add URL line breaks if available
\urlstyle{same}
\hypersetup{
  pdftitle={L'assurance-maladie obligatoire suisse dans l'impasse. Une analyse du compromis socio-politique derrière la Loi fédérale sur l'assurance-maladie (LAMal)},
  pdfauthor={Celâl Güney; Gaia Valenti},
  colorlinks=true,
  linkcolor={blue},
  filecolor={Maroon},
  citecolor={Blue},
  urlcolor={Blue},
  pdfcreator={LaTeX via pandoc}}



\draftfalse

\begin{document}
\title{L'assurance-maladie obligatoire suisse dans l'impasse. Une
analyse du compromis socio-politique derrière la Loi fédérale sur
l'assurance-maladie (LAMal)}

\authors{Celâl Güney\affil{1}, Gaia Valenti\affil{1}}
\affiliation{1}{unige, }\affiliation{2}{Université de Genève, }
\correspondingauthor{Celâl Güney}{celal.gueney@unige.ch}







\section{Annexe}\label{annexe}

\subsection{Stratégie empirique pour les modèles de la section
4}\label{stratuxe9gie-empirique-pour-les-moduxe8les-de-la-section-4}

La spécification des modèles présentés dans la section 4 a pour objectif
de reproduire une analyse néoréaliste des facteurs influençant les
attentes sociales. Les variables dépendantes sont binarisées afin de
conduire des estimations de modèles logistiques binaires en utilisant
l'estimation du maximum de vraisemblance. La variable d'opinion sur
l'augmentation de la franchise minimale de l'assurance-maladie de base
est construite sur une échelle Likert de 1 (fortement contre) à 5
(fortement pour). Nous avons créé une variable dichotomique pour les
deux premières catégories (fortement contre et contre). Pour le choix du
vote pour l'initiative d'allègement des primes, nous avons dichotomisé
en binarisant le vote en faveur de l'initiative. Les principales
variables explicatives que nous avons considérées sont un indice de
classe sociale mesurée par la classification européenne des groupes
socioéconomiques (ESEG), le revenu mensuel brut du ménage, à partir
duquel nous calculons le niveau de revenu par décile, ainsi qu'une
batterie d'indicateurs socio-démographiques (âge, langue, cantons,
genre) et idéologiques (auto-positionnement gauche-droite, vote,
probabilité de vote pour les principaux partis politiques, opinions sur
des sujets économiques et identitaires). Pour chacune de nos deux
variables dépendantes, nous avons procédé à une sélection de modèle pas
à pas descendante avec le critère d'information d'Akaike (AIC). Cette
procédure a abouti au modèle suivant pour l'opposition à l'augmentation
de la franchise minimale:

\begin{equation}\phantomsection\label{eq-1}{
P(y_i = 1) = logit^{-1}(\beta_0 + \beta_{d}D_i + \beta X_i + \gamma_p P_i + \eta_o O_i + \epsilon_i) 
}\end{equation}

Avec \(P(y_i = 1)\) la probabilité de l'individu \(i\) d'être contre
l'augmentation de la franchise minimale, \(D_i\) le niveau de décile
auquel appartient l'individu \(i\), \(X_i\) un vecteur de
caractéristiques individuelles socio-démographique (âge, genre, langue),
\(P_i\) un vecteur de variable sur la probabilité de voter pour certains
partis politiques suisses (Le Centre, le Parti Socialiste, les Verts et
les Verts Libéraux) sur une échelle de 0 à 10, et \(O_i\) l'opinion sur
une série d'enjeux comme l'intégration européenne, le salaire minimum ou
encore les dépenses sociales. \(\epsilon_i\) est le terme d'erreur. De
manière surprenante, la variable pour les groupes socio-économiques ESEG
n'a pas été retenue par la procédure de sélection, ce qui implique que
la classe sociale telle que mesurée par l'indice ESEG ne semble pas
jouer un rôle déterminant par rapport au niveau de revenu et aux autres
variables de contrôle Comme le schéma de classe ESEG est une variable
importante pour notre analyse, nous considérons quand même les résultats
d'une régression simple de nos deux variables dépendantes sur les
catégories socio-professionnelles ESEG (résultats disponibles dans
l'annexe). Encore plus étonnant, la variable contrôlant pour les cantons
n'a pas non plus été retenue, ce qui suggère que l'effet d'appartenance
à un canton est négligeable. Le coefficient de corrélation interclasse
(ICC) calculé à partir d'un modèle multiniveau avec les cantons en tant
que niveau confirme que les variations entre cantons sont faibles (voir
annexe). Le modèle pour le vote en faveur de l'initiative pour
l'allègement des primes prend une forme similaire à (1), avec quelques
différences au niveau des variables retenues car certaines d'entre elles
n'étaient pas disponible dans la quatrième vague. Comme pour le premier
modèle, le schéma de classe ESEG n'a pas été retenu par la procédure de
sélection, ni le canton de résidence.

\begin{table}

\caption{\label{tbl-votations}Votations en lien avec le système de santé
en Suisse. Source: Swissvotes}

\centering{

\fontsize{12.0pt}{14.0pt}\selectfont
\begin{tabular*}{\linewidth}{@{\extracolsep{\fill}}rlrr}
\toprule
Date & Votation & Résultat & \% Oui \\ 
\midrule\addlinespace[2.5pt]
30.07.1882 & Loi sur les épidémies & 0 & 21.10 \\ 
26.10.1890 & Droit de légiférer sur l'assurance en cas d'accident et de maladie & 1 & 75.44 \\ 
20.05.1900 & Loi sur l'assurance maladie, accidents et militaire & 0 & 30.21 \\ 
04.02.1912 & Loi sur l'assurance en cas de maladie et d'accidents & 1 & 54.36 \\ 
04.05.1913 & Lutte contre les maladies de l'homme et des animaux & 1 & 60.32 \\ 
22.05.1949 & Loi sur la lutte contre la tuberculose & 0 & 24.85 \\ 
08.12.1974 & Initiative «pour une meilleure assurance-maladie» & 0 & 26.71 \\ 
08.12.1974 & Contre-projet à l'initiative «pour une meilleure assurance-maladie» & 0 & 31.84 \\ 
10.03.1985 & Suppression de l'obligation de la Confédération d'allouer des subventions dans le domaine de la santé publique & 1 & 52.99 \\ 
06.12.1987 & Loi sur l'assurance-maladie & 0 & 28.72 \\ 
16.02.1992 & «Initiative des caisses-maladie» & 0 & 39.27 \\ 
26.09.1993 & Mesures concernant l'assurance-maladie & 1 & 80.55 \\ 
04.12.1994 & Loi sur l'assurance-maladie & 1 & 51.80 \\ 
04.12.1994 & Initiative «pour une saine assurance-maladie» & 0 & 23.45 \\ 
26.11.2000 & Initiative «pour des coûts hospitaliers moins élevés» & 0 & 17.89 \\ 
09.02.2003 & Loi sur les participations cantonales aux coûts des traitements hospitaliers & 1 & 77.36 \\ 
18.05.2003 & «Initiative-santé» & 0 & 27.09 \\ 
11.03.2007 & Initiative pour une caisse maladie unique & 0 & 28.76 \\ 
01.06.2008 & Article constitutionnel sur l'assurance-maladie & 0 & 30.52 \\ 
17.05.2009 & Article constitutionnel sur les médecines complémentaires & 1 & 67.03 \\ 
17.06.2012 & Loi sur l'assurance-maladie (Réseaux de soins) & 0 & 23.95 \\ 
22.09.2013 & Loi sur les épidémies & 1 & 60.00 \\ 
09.02.2014 & Initiative «Financer l'avortement est une affaire privée» & 0 & 30.18 \\ 
18.05.2014 & Arrêté fédéral concernant les soins médicaux de base & 1 & 88.07 \\ 
28.09.2014 & Initiative «Pour une caisse publique d'assurance-maladie» & 0 & 38.16 \\ 
28.11.2021 & Initiative sur les soins infirmiers & 1 & 60.98 \\ 
15.05.2022 & Principe du consentement présumé pour le don d’organes & 1 & 60.20 \\ 
09.06.2024 & Initiative d’allègement des primes & 0 & 44.53 \\ 
09.06.2024 & Initiative pour un frein aux coûts dans le système de santé & 0 & 37.23 \\ 
24.11.2024 & Financement uniforme des prestations ambulatoires et stationnaires & 1 & 53.31 \\ 
\bottomrule
\end{tabular*}

}

\end{table}%

\textsubscript{Source:
\href{https://Jeylal.github.io/annexe-assurance-maladie/index.qmd.html}{Article
Notebook}}

\subsection{Matériel
supplémentaire}\label{matuxe9riel-suppluxe9mentaire}

\textsubscript{Source:
\href{https://Jeylal.github.io/annexe-assurance-maladie/index.qmd.html}{Article
Notebook}}

\textsubscript{Source:
\href{https://Jeylal.github.io/annexe-assurance-maladie/index.qmd.html}{Article
Notebook}}

\textsubscript{Source:
\href{https://Jeylal.github.io/annexe-assurance-maladie/index.qmd.html}{Article
Notebook}}

\textsubscript{Source:
\href{https://Jeylal.github.io/annexe-assurance-maladie/index.qmd.html}{Article
Notebook}}

\begin{table}

\caption{\label{tbl-skim1}Statistiques descriptive variables du premier
modèle}

\centering{

}

\end{table}%

\textsubscript{Source:
\href{https://Jeylal.github.io/annexe-assurance-maladie/index.qmd.html}{Article
Notebook}}

\phantomsection\label{cell-fig-likert1}
\begin{figure}[H]

\centering{

\pandocbounded{\includegraphics[keepaspectratio]{index_files/figure-pdf/fig-likert1-1.pdf}}

}

\caption{\label{fig-likert1}Statistiques descriptives des variables à
échelle Likert du premier modèle}

\end{figure}%

\textsubscript{Source:
\href{https://Jeylal.github.io/annexe-assurance-maladie/index.qmd.html}{Article
Notebook}}

\phantomsection\label{cell-fig-likert2}
\begin{figure}[H]

\centering{

\pandocbounded{\includegraphics[keepaspectratio]{index_files/figure-pdf/fig-likert2-1.pdf}}

}

\caption{\label{fig-likert2}Statistiques descriptives des variables à
échelle Likert du premier modèle (2)}

\end{figure}%

\textsubscript{Source:
\href{https://Jeylal.github.io/annexe-assurance-maladie/index.qmd.html}{Article
Notebook}}

\phantomsection\label{cell-fig-likert3}
\begin{figure}[H]

\centering{

\pandocbounded{\includegraphics[keepaspectratio]{index_files/figure-pdf/fig-likert3-1.pdf}}

}

\caption{\label{fig-likert3}Statistiques descriptives des variables à
échelle Likert du premier modèle (3)}

\end{figure}%

\textsubscript{Source:
\href{https://Jeylal.github.io/annexe-assurance-maladie/index.qmd.html}{Article
Notebook}}

\textsubscript{Source:
\href{https://Jeylal.github.io/annexe-assurance-maladie/index.qmd.html}{Article
Notebook}}

\phantomsection\label{cell-fig-likert4}
\begin{figure}[H]

\centering{

\pandocbounded{\includegraphics[keepaspectratio]{index_files/figure-pdf/fig-likert4-1.pdf}}

}

\caption{\label{fig-likert4}Statistiques descriptives des variables à
échelle Likert du deuxième modèle (3)}

\end{figure}%

\textsubscript{Source:
\href{https://Jeylal.github.io/annexe-assurance-maladie/index.qmd.html}{Article
Notebook}}

\begin{table}

\caption{\label{tbl-reglogit}Régressions logit (log(odd))}

\centering{

\centering
\begin{talltblr}[         %% tabularray outer open
entry=none,label=none,
note{}={+ p \num{< 0.1}, * p \num{< 0.05}, ** p \num{< 0.01}, *** p \num{< 0.001}},
]                     %% tabularray outer close
{                     %% tabularray inner open
colspec={Q[]Q[]},
hline{2}={1-2}{solid, black, 0.05em},
hline{42}={1-2}{solid, black, 0.05em},
hline{1}={1-2}{solid, black, 0.08em},
hline{48}={1-2}{solid, black, 0.08em},
column{2}={}{halign=c},
column{1}={}{halign=l},
}                     %% tabularray inner close
& Augmentation franchise minimale \\
(Intercept) & \num{0.027} \\
& (\num{0.310}) \\
income\_adj\_decile & \num{-0.033}*** \\
& (\num{0.010}) \\
op\_limit\_immigration & \num{-0.082}** \\
& (\num{0.030}) \\
op\_protec\_env & \num{-0.122}*** \\
& (\num{0.035}) \\
op\_social\_expenses & \num{0.155}*** \\
& (\num{0.034}) \\
op\_eu\_integration & \num{0.074}** \\
& (\num{0.027}) \\
op\_chances\_foreigners & \num{0.036} \\
& (\num{0.026}) \\
op\_gender\_equality & \num{0.066}+ \\
& (\num{0.035}) \\
op\_taxes\_high\_income & \num{-0.071}* \\
& (\num{0.032}) \\
op\_min\_wage & \num{0.147}*** \\
& (\num{0.034}) \\
op\_incr\_retirementAge & \num{-0.335}*** \\
& (\num{0.031}) \\
probs\_vote\_centre & \num{-0.028}** \\
& (\num{0.010}) \\
probs\_vote\_ps & \num{0.048}*** \\
& (\num{0.012}) \\
probs\_vote\_verts & \num{-0.024}+ \\
& (\num{0.014}) \\
probs\_vote\_vertsliberaux & \num{-0.036}** \\
& (\num{0.011}) \\
langueFrench & \num{0.467}*** \\
& (\num{0.069}) \\
langueItalian & \num{0.172} \\
& (\num{0.154}) \\
langueRomansh & \num{0.004} \\
& (\num{0.545}) \\
W1\_age & \num{0.010}*** \\
& (\num{0.002}) \\
genderFemale & \num{0.290}*** \\
& (\num{0.057}) \\
Num.Obs. & \num{6280} \\
AIC & \num{7950.8} \\
BIC & \num{8085.7} \\
Log.Lik. & \num{-3955.407} \\
F & \num{30.731} \\
RMSE & \num{0.47} \\
\end{talltblr}

}

\end{table}%

\textsubscript{Source:
\href{https://Jeylal.github.io/annexe-assurance-maladie/index.qmd.html}{Article
Notebook}}

\begin{table}
\centering
\begin{talltblr}[         %% tabularray outer open
entry=none,label=none,
note{}={+ p \num{< 0.1}, * p \num{< 0.05}, ** p \num{< 0.01}, *** p \num{< 0.001}},
]                     %% tabularray outer close
{                     %% tabularray inner open
colspec={Q[]Q[]},
hline{2}={1-2}{solid, black, 0.05em},
hline{40}={1-2}{solid, black, 0.05em},
hline{1}={1-2}{solid, black, 0.08em},
hline{46}={1-2}{solid, black, 0.08em},
column{2}={}{halign=c},
column{1}={}{halign=l},
}                     %% tabularray inner close
& Initiative pour l'allègement des primes \\
(Intercept) & \num{-1.112}** \\
& (\num{0.425}) \\
income\_adj\_decile & \num{-0.124}*** \\
& (\num{0.016}) \\
I(eseg10 == "Retired")TRUE & \num{0.553}*** \\
& (\num{0.137}) \\
lr & \num{-0.076}** \\
& (\num{0.027}) \\
op\_state\_intervention & \num{-0.266}*** \\
& (\num{0.051}) \\
op\_2many\_worries\_abt\_env\_vs\_prices & \num{0.094}* \\
& (\num{0.043}) \\
op\_min\_wage & \num{0.336}*** \\
& (\num{0.053}) \\
op\_incr\_retirement\_age & \num{-0.128}* \\
& (\num{0.050}) \\
op\_foreigners\_votingrights & \num{0.163}** \\
& (\num{0.052}) \\
vote\_choiceThe Centre (former CVP/ BDP) & \num{-0.057} \\
& (\num{0.163}) \\
vote\_choiceSP/PS - Social Democratic Party & \num{0.806}*** \\
& (\num{0.180}) \\
vote\_choiceSVP/UDC - Swiss People's Party & \num{0.134} \\
& (\num{0.160}) \\
vote\_choiceGPS/PES - Green Party & \num{0.669}** \\
& (\num{0.216}) \\
vote\_choiceGLP/PVL - Green Liberal Party & \num{0.052} \\
& (\num{0.196}) \\
vote\_choiceOther party, several/all parties & \num{0.516}* \\
& (\num{0.222}) \\
genderFemale & \num{-0.376}*** \\
& (\num{0.088}) \\
W4\_age & \num{0.013}*** \\
& (\num{0.004}) \\
languageFrench & \num{0.504}*** \\
& (\num{0.111}) \\
languageItalian & \num{0.929}*** \\
& (\num{0.225}) \\
Num.Obs. & \num{2999} \\
AIC & \num{3370.0} \\
BIC & \num{3484.1} \\
Log.Lik. & \num{-1666.006} \\
F & \num{30.126} \\
RMSE & \num{0.43} \\
\end{talltblr}
\end{table}

\textsubscript{Source:
\href{https://Jeylal.github.io/annexe-assurance-maladie/index.qmd.html}{Article
Notebook}}

\textsubscript{Source:
\href{https://Jeylal.github.io/annexe-assurance-maladie/index.qmd.html}{Article
Notebook}}

\begin{table}

\caption{\label{tbl-canton}Modèle multiniveau: cantons suisses en
niveau}

\centering{

\centering
\begin{tblr}[         %% tabularray outer open
]                     %% tabularray outer close
{                     %% tabularray inner open
colspec={Q[]Q[]Q[]},
hline{2}={1-3}{solid, black, 0.05em},
hline{5}={1-3}{solid, black, 0.05em},
hline{1}={1-3}{solid, black, 0.08em},
hline{12}={1-3}{solid, black, 0.08em},
column{2-3}={}{halign=c},
column{1}={}{halign=l},
}                     %% tabularray inner close
& Augmentation franchise minimale & Initiative pour l'allègement des primes \\
(Intercept) & \num{-0.301} & \num{0.327} \\
& (\num{0.090}) & (\num{0.067}) \\
SD (Intercept canton) & \num{0.377} & \num{0.294} \\
Num.Obs. & \num{3608} & \num{8110} \\
R2 Marg. & \num{0.000} & \num{0.000} \\
R2 Cond. & \num{0.041} & \num{0.026} \\
AIC & \num{4848.6} & \num{10964.8} \\
BIC & \num{4861.0} & \num{10978.8} \\
ICC & \num{0.0} & \num{0.0} \\
RMSE & \num{0.49} & \num{0.49} \\
\end{tblr}

}

\end{table}%

\textsubscript{Source:
\href{https://Jeylal.github.io/annexe-assurance-maladie/index.qmd.html}{Article
Notebook}}

\begin{table}

\caption{\label{tbl-reg2}Régressions logistiques simples: opposition à
l'augmentation de la franchise et vote pour l'allègement des primes sur
les catégories socio-professionnelles ESEG (Log(Odd))}

\centering{

\centering
\begin{talltblr}[         %% tabularray outer open
entry=none,label=none,
note{}={+ p \num{< 0.1}, * p \num{< 0.05}, ** p \num{< 0.01}, *** p \num{< 0.001}},
]                     %% tabularray outer close
{                     %% tabularray inner open
colspec={Q[]Q[]Q[]},
hline{2}={1-3}{solid, black, 0.05em},
hline{24}={1-3}{solid, black, 0.05em},
hline{1}={1-3}{solid, black, 0.08em},
hline{29}={1-3}{solid, black, 0.08em},
column{2-3}={}{halign=c},
column{1}={}{halign=l},
}                     %% tabularray inner close
& Augmentation franchise minimale & Initiative pour l'allègement des primes \\
(Intercept) & \num{-0.052} & \num{-0.984}*** \\
& (\num{0.114}) & (\num{0.179}) \\
eseg10Clerks and skilled service employees & \num{0.462}** & \num{0.452}* \\
& (\num{0.143}) & (\num{0.229}) \\
eseg10Lower status employees & \num{0.452}** & \num{0.609}* \\
& (\num{0.166}) & (\num{0.265}) \\
eseg10Managers & \num{0.092} & \num{0.094} \\
& (\num{0.134}) & (\num{0.215}) \\
eseg10Professionals & \num{0.293}* & \num{0.454}* \\
& (\num{0.125}) & (\num{0.195}) \\
eseg10Retired & \num{0.505}*** & \num{1.131}*** \\
& (\num{0.124}) & (\num{0.188}) \\
eseg10Skilled industrial employees & \num{0.589}*** & \num{0.485}+ \\
& (\num{0.168}) & (\num{0.281}) \\
eseg10small entrepreneurs & \num{0.369}+ & \num{0.504} \\
& (\num{0.197}) & (\num{0.307}) \\
eseg10Student & \num{0.160} & \num{0.449}+ \\
& (\num{0.142}) & (\num{0.266}) \\
eseg10Technicians and associated professionals employees & \num{0.391}** & \num{0.436}* \\
& (\num{0.135}) & (\num{0.216}) \\
eseg10Unemployed or disabled & \num{0.573}*** & \num{0.548}* \\
& (\num{0.149}) & (\num{0.249}) \\
Num.Obs. & \num{7535} & \num{3392} \\
AIC & \num{10241.2} & \num{4511.7} \\
BIC & \num{10317.4} & \num{4579.2} \\
Log.Lik. & \num{-5109.620} & \num{-2244.872} \\
RMSE & \num{0.49} & \num{0.48} \\
\end{talltblr}

}

\end{table}%

\textsubscript{Source:
\href{https://Jeylal.github.io/annexe-assurance-maladie/index.qmd.html}{Article
Notebook}}

\pandocbounded{\includegraphics[keepaspectratio]{index_files/figure-pdf/unnamed-chunk-4-1.pdf}}

\pandocbounded{\includegraphics[keepaspectratio]{index_files/figure-pdf/unnamed-chunk-4-2.pdf}}

\textsubscript{Source:
\href{https://Jeylal.github.io/annexe-assurance-maladie/index.qmd.html}{Article
Notebook}}

\pandocbounded{\includegraphics[keepaspectratio]{index_files/figure-pdf/unnamed-chunk-5-1.pdf}}

\textsubscript{Source:
\href{https://Jeylal.github.io/annexe-assurance-maladie/index.qmd.html}{Article
Notebook}}




\end{document}
